\documentclass{article}

% Language setting
% Replace `english' with e.g. `spanish' to change the document language
\usepackage[english]{babel}

% Set page size and margins
% Replace `letterpaper' with `a4paper' for UK/EU standard size
\usepackage[a4paper,top=2cm,bottom=2cm,left=3cm,right=3cm,marginparwidth=1.75cm]{geometry}

% Useful packages
\usepackage{amsmath}
\usepackage{graphicx}
\usepackage[colorlinks=true, allcolors=blue]{hyperref}
\usepackage{cleveref}

\usepackage[
  autocite    = superscript,
  backend     = bibtex,
  sortcites   = true,
%   style       = numeric,
  doi=false, url=true, isbn=false,
  maxcitenames=2, maxbibnames=2
  ]{biblatex}

\bibliography{references}

\renewcommand{\vec}[1]{\boldsymbol{#1}}
% \renewcommand{\hat}[1]{\hat{\vec {#1}}}

\title{Effective dynamics of an aircraft}

\author{Art L. Gower, Matheus de Carvalho Loures, Paulo Piva}

\begin{document}
\maketitle

\begin{abstract}
    To plan optimal aircraft routes we can not completely solve and describe 3D fluid structure interaction. This would be computationally very intense and unnecessary. Instead, we need to capture the main features related to fuel consumption, ability to turn, and affects from the atmosphere such as drag and lift. To achieve this we develop effective dynamical equations, together with a simple numerical scheme to solve these equations. The method is simple enough to using nonlinear optimisation to plan routes with minimal fuel, or distance, or time.  
\end{abstract}

\section{Equations of motion}
The main forces acting on the aircraft are due to thrust, drag, lift, and a turning force. More accurately: an aircraft turns by rolling and then using lift, however we will not model these details, and instead have a turning force which is similar to a lift force.See \cref{fig:plane-sketch} for a sketch. 


\begin{figure}[ht]
    \centering
    \includegraphics[width = 0.4\linewidth]{"plane-sketch.png"}
    \caption{A sketch showing that the aircraft velocity vector is given by $\vec v$ (relative to the ground), and the velocity of the wind (relative to the ground) is given by $\vec u$. 
    }
    \label{fig:plane-sketch}
\end{figure}

% As it is impractical to solve for even the rigid body dynamics of the aircraft, which would allow for rotations, we assume that the aircraft always points in the direction of travel. 
To describe the aircraft dynamics it is useful to use a local coordinate system with basis vectors:
\begin{align}
    & \vec e_a \quad \text{(aligned with the direction of travel)}
    \\
    & \vec e_r \quad \text{(radial direction from earth centre to aircraft)}
    \\
    & \vec e_p = \vec e_r \times \vec e_v  \quad \text{(perpendicular to travel direction)}
\end{align}
where we will only model the 2D dynamics and assume the altitude is fixed, so that $\vec e_a$ is always orthogonal to $\vec e_r$. Note that changes of altitude based on balance of lift with gravity can easily be accommodated without modelling the motion that leads to that change. 

There are two main forces on the aircraft, those that can be controlled by the pilot $\vec f$ and those that external $\vec w$. The forces that can be controlled are: 
\begin{equation}
    \vec{f} =  T(\dot{m}) \vec{e}_a + {L_p}(\vec v - \vec u, \alpha) \vec{e}_p  + {L_r}(\vec v - \vec u,\beta) \vec{e}_r,
\end{equation}
where $\dot m$ is the rate of change of mass in time (from using fuel), $v = |\vec v|$, $T(\dot{m})$ is the force from thrust, ${L_p}(v, \alpha)$ is a force which leads to turns in the $\vec e_p$ direction, where $\alpha$ captures the amount the pilot tries to turn, and $L_r(v,\beta)$ are the lift forces, where $\beta$ is the amount to pilot tries to lift. 

A simple and effective choice for thrust is
\begin{equation}
    T(\dot{m}) = - \dot m C_{T},
    %  \quad \text{and} \quad 
    % L_p (v, \alpha) = v^2 \alpha,
\end{equation}
where $C_T$ is a constant which describes how efficiently burning fuel $\dot m$ is converted into a thrust. More accurately, 
\[
C_T = \text{(exhaust velocity)} - \text{(relative airspeed)},
\] 
for subsonic flight \cite[Chapter 4]{anderson2005introduction}, where exhaust velocity is the speed of exhaust relative to the aircraft, and relative airspeed is equal to $|\vec v - \vec u|$.

A simple and effective formula for the turning force is
\begin{equation}
    L_p (v, \alpha) = |(\vec v - \vec u) \cdot \vec e_a|^2 \alpha.
\end{equation}
In practive, turning is due to rolling and then lift. The forces that lead to rolling and lift are due to wind drag, which is proportional to the relative airspeed in the direction of travel. The variable $\alpha$ is bounded: $\alpha \in [-C_\alpha, C_\alpha]$, where the positive constant $C_\alpha$ depends on the type of aircraft.

The dynamics of the aircraft are now governed by balance of momentum $\vec p = m \vec v$:
% \begin{equation}
%     \vec{f}= \vec{w}(r,\vec{v},t) - T(\dot{m},t)\vec{e}_v+ \vec{L_c}(v(t), \alpha(t)) +\vec{L_u}(\vec{v}(t),\beta(t))
% \end{equation}
% We start supposing the the plane can only move in a spherical shell with coordinates $(\theta, \phi)$. We will write the problem in the coordinate of the perpendicular direction of the velocity ($\vec{e}_p$) and velocity direction ($\vec{e}_v$) in the spherical shell.
% The Thrust force in general coordinates is:
% \begin{equation}
%   - T(\dot{m},t) \vec{e}_v(t)
% \end{equation}
% observe that the normal to $\vec{e}_v(t)$ is $\vec{e}_p(t)=\vec{e}_r(t)\times  \vec{e}_v(t)$ in the sphere surface. The normal component normal to the sphere surface is $\hat{e}_r$. The force to turn on the surface is then
% \begin{equation}
%     \vec{L}_c(\alpha(t),t)=L_c(\alpha(t),v(t))\vec{e}_p(t)
% \end{equation}
% while the lift force to turn in the radial direction is 
% \begin{equation}
%     \vec{L}_u(\alpha(t),t)=L_u(\alpha(t),v(t))\vec{e}_r(t)
% \end{equation}
% So $\vec{f}$ is given by:
% \begin{equation}
%     \vec{f}= \vec{w}(r,t)   - T(\dot{m},v(t)) \vec{e}_v(t)+ L_c(\alpha(t),v(t))\vec{e}_p(t) + L_u(\alpha(t),v(t))(\vec{e}_r(t)))
% \end{equation}
% Now we write the movement equations on the there directions ($\vec{e}_v(t), \hat n(t),\vec{e}_r(t)$)
\begin{align}
    % \frac{dp}{dt}= \vec{f} \implies 
   \notag & \frac{d}{dt} \vec p = \vec{f} + \vec{w} \implies
\\ \label{eqns:motion-vector-3D}
    & \dot{\vec p} = T(\dot{m}) \vec{e}_a + {L_p}(\vec v - \vec u, \alpha) \vec{e}_p  + {L_r}(\vec v - \vec u,\beta) \vec{e}_r + \vec w,
% \implies
% \\
%     & \dot{m}(t)v(t)\vec{e}_v(t) +m(t)\dot v(t)\vec{e}_v(t) + m(t) v(t)\dot{\vec{e}_v}(t) = \vec{f}
\end{align}
where $\vec w$ are the forces due to external factors, such as the wind and altitude. It is generally a function of the relative airspeed $\vec v - \vec u$, however we do not need to give explicit forms for this forces.

For the lift force $L_r$, we assume it is enough to balance the external vertical, or radial, forces such as gravity. That is, by taking the dot product of $\vec e_r$ on the right side of \eqref{eqns:motion-vector} we reach that 
\[
\vec w \cdot \vec e_r = -  {L_r}(\vec v - \vec u,\beta).
\]
The simplest way to enforce this is to write the external forces in the form:
\[
\vec w = w_a \vec e_a + w_p \vec e_p + w_r \vec e_r,
\]
then the two equations above substituted into \eqref{eqns:motion-vector-3D} lead to
\begin{align}
   \label{eqns:motion-vector}
    & \dot{\vec p} = (T(\dot{m}) + w_a) \vec{e}_a + ({L_p}(\vec v - \vec u, \alpha) + w_p) \vec{e}_p.
% \implies
% \\
%     & \dot{m}(t)v(t)\vec{e}_v(t) +m(t)\dot v(t)\vec{e}_v(t) + m(t) v(t)\dot{\vec{e}_v}(t) = \vec{f}
\end{align}
% \begin{equation}
%     m \dot{\vec{v}} \cdot \vec e_r = - \dot{m}\vec{v} \cdot \vec e_r.
% \end{equation}
Equation \eqref{eqns:motion-vector} can be used to update the momentum $\vec p$ in time, which we show in more detail in the next section. 

Finally, we need to calculate how the plane's orientation $\vec e_a$ changes. If the pilot does not actively turn the plan and $L_p = 0$, then the plane will change its orientation with the flow. In others words, 
\[
\frac{d}{dt} \left[ \vec e_a \cdot (\vec v - \vec u) \right] =0 \implies  \dot {\vec e}_a \cdot (\vec v - \vec u) = - \vec e_a \cdot(\dot {\vec v} - \dot{\vec u}),
\]
which can be used to update the direction $\vec e_a$.


\textbf{NOTE: Below is a work in progress. In fact, it is not clear the aircraft's orientation follows the flow as shown above. For example a spherical rock would not change its orientation with the flow, but a paper airplane would. The aircraft is probably more like a paper airplane.}

Turning the plane also changes its orientation, which as a consequence changes the basis vectors $\vec e_a$ and $\vec e_p$. We assume that there is a linear relationship between turning force $L_p$ and the torque force that causes the plane to rotate. Then, depending on the moment of inertia of aircraft, this torque will cause a rotation. To calculate this rotation, let choose a position $\vec x$ for the aircraft in a spherical coordinate system $(r,\theta,\phi)$ with
\begin{equation}
    \vec x = r [ \sin \phi \cos \theta, \sin \phi \sin \theta,   \cos \phi].
\end{equation}
Then we can write the orientation of the aircraft in the form
\[
\vec e_a = \cos \tau \vec e_\theta + \sin \tau \vec e_\phi,
\]
where we call $\tau$ the angle of orientation of the aircraft, and note that the basis vectors $\vec e_\theta$ and $\vec e_\phi$ can be defined from:
\begin{equation}
 \frac{d }{d t}\vec x = \vec v = r \sin \phi  {\vec e_\theta} \dot \theta + r  {\vec e_\phi} \dot \phi.
\end{equation} 

Now the rotation of the plane is given from
\begin{equation}
    \ddot \tau = C_\tau L_p \alpha,
\end{equation}
for some constant $C_\tau$ which depends on the type of aircraft. 

% To capture this, we note that in the absence of external forces, such as wind, the plane would point in the direction of travel. We also assume that that external forces do not make the plane rotate, this is reasonable, as a very high vorticity wind would be needed to substantially rotate the plane. Therefore, we can calculate the new orientation of the aircraft  


\subsection{Forward marching numerical method}

Then the velocity vector of the aircraft is given by $\dot{\vec x} = \vec v$, with $\vec v$. Rewriting in spherical coordinates we get
\[
\vec v = r \sin \phi  {\vec e_\theta} \dot \theta + r  {\vec e_\phi} \dot \phi, 
\]
by assuming that the altitude $r$ is fixed. Then, from the above we deduce that
\begin{equation} \label{eqn:update_thetaphi}
    r \sin \phi \dot \theta = \vec v \cdot {\vec e}_\theta  \quad \text{and} \quad 
r \dot \phi = \vec v \cdot {\vec e}_\phi.
\end{equation}
From the above we can see that it is convenient to write the components $\vec v$ in terms of the local coordinate system ${\vec e}_\theta, {\vec e}_\phi, {\vec e}_r$ coordinate system. That is, in the code, 
\[
\vec v[1] = \vec v \cdot {\vec e}_\theta, \quad 
\vec v[2] = \vec v \cdot {\vec e}_\phi, 
\quad 
\vec v[3] = \vec v \cdot {\vec e}_r.
\]
We can then use the equations \eqref{eqn:update_thetaphi}  to calculate $\theta(t+h)$ and $\phi(t+h)$ by substituting 
\[
\dot \theta = \frac{ \theta(t+h) - \theta(t)}{h} \quad \text{and} \quad 
\dot \phi = \frac{ \phi(t+h) - \phi(t)}{h},
\]
into \eqref{eqn:update_thetaphi} and solving for $\theta(t+h)$ and $\phi(t+h)$. For consistency, we have that the components of all vectors are given in terms of the local basis in the code.

The equations of motion (\ref{eqn:thrust-momentum} - \ref{eqn:lift-momentum}) can now be turned into a forward marching numerical method to predict the trajectory of the aircraft, which we briefly summarise below.

Equation \eqref{eqn:thrust-momentum} can be used to update the speed $v(t)$ by substituting
\[
\dot { v}(t) = \frac{{v}(t+h) - {v}(t)}{h},
\]
and then solving for ${v}(t+h)$. From the second equation \eqref{eqn:turn-momentum} we can update the direction $\vec {e}_v$ and as a consequence $\vec e_p$. To start
\[
\dot{\vec e}_v = a {\vec e}_p + b {\vec e}_r, 
\]
because $\dot {\vec e}_v$ is orthogonal to ${\vec e}_v$. Substituting the above into \eqref{eqn:turn-momentum} then leads to
\begin{equation}
 a =  {\vec{e}}_p \cdot \dot{\vec{e}}_v =  (w_n(r,v,t) + L_c(\alpha(t),v(t) ) / (m(t) v(t)).
\end{equation}
To obtain $b$ we use 
\begin{equation}
b = \dot{\vec e}_v \cdot  {\vec e}_r = - {\vec e}_v \cdot  \dot {\vec e}_r =  - {\vec e}_v \cdot  {\vec v} / r = - v / r,    
\end{equation}
where we also used the $\vec e_r = \vec r / r \implies \dot{\vec e}_r = \vec v / r$ for fixed $r$.



% \eqref{eqn:turn-momentum} we can update $\vec {e}_p$ and as a consequence $\vec e_v$. To start
% \[
% \dot{\vec e}_p = a {\vec e}_v + b {\vec e}_r, 
% \]
% because $\dot {\vec e}_p$ is orthogonal to ${\vec e}_p$. Equation \eqref{eqn:turn-momentum} gives us 
% \begin{equation}
%  a =  \dot{\vec{e}}_p \cdot \vec{e}_v = - (w_n(r,v,t) + L_c(\alpha(t),v(t) ) / (m(t) v(t)).
% \end{equation}
% To obtain $b$ we use 
% \begin{equation}
% b = \dot{\vec e}_p \cdot  {\vec e}_r = - {\vec e}_p \cdot  \dot {\vec e}_r =  - {\vec e}_p \cdot  {\vec v} / r = 0,    
% \end{equation}
% which is further explained below. 
% \[
% \vec{x}=r{\vec{e}}_r \Rightarrow \vec{v} =\dot{\vec{x}}=\dot{r}{\vec{e}}_r +r \dot{\vec{e}}_r \Rightarrow \dot{\vec{e}}_r=\frac{\vec{v}}{r} 
% \]
 
To summarise we can use the above to update the direction $\vec e_v$
\begin{equation}
    \vec e_v(t+h) = \vec e_v(t) + a h \vec e_p(t) - \frac{v h}{r} \vec e_r(t)
\end{equation}
where using the coordinate system $(\theta,\phi,r)$ we always have that $\vec e_r = [0,0,1]$. As we have made first order approximations in $h$ the norm of $\vec e_v(t+h)$ will only be approximately 1. It is better to correct this by taking $ \vec e_v(t+h)  \leftarrow  \vec e_v(t+h) / | \vec e_v(t+h) |$. Finally, we can then update the perpendicular direction:
\begin{equation}
    %    \quad \text{and} \quad 
    \vec e_p(t+h) = \vec e_r \times \vec e_v(t+h).
\end{equation}

%  and $\vec e_p$ through

% \textbf{Art: I suppose we have to normalise $\vec e_p(t+h) $ so that it's magnitude does not stray over time?}


\section{Finding the right altitude to keep a certain speed above the tropopause}

The tropopause is the altitude that separates the troposphere and the stratosphere 

The force balance in normal direction gives equilibrium of the plane at a constant altitude $H$. 
The air density at a certain altitude can be approximated as

\begin{equation}
    \rho= \rho_{\text{trop}}e^{-\left( \frac{g}{RT_{\text{trop}}}(H-h_{\text{trop})} \right)},
\end{equation}
where the sub-index 'trop' means the tropopause value of the density,temperature, and altitude provided in the NATS guide. The pressure also follows the same behaviour.

\begin{equation}
    P= P_{\text{trop}}\mathrm{e}^{-\left( \frac{g}{RT_{\text{trop}}}(H-h_{\text{trop})} \right)}
\end{equation}
The lift force is calculated as

\begin{equation}
    L_u= \frac{\beta C_L}{2} \rho u^2S
\end{equation}
Where $u=|\vec{v}-\vec{W}|$ is the speed of the plane related to the wind, m is the mass of the plane, g is gravity , S is the area of the wing  projected at the plane normal to $\vec{e}_r$, and $C_L$ is the lift coefficient with $\beta$ being its control parameter. From that we can see that the lift force is proportional to the density.  So the Lift force is 

\begin{equation}
    L_u= \frac{\beta C_L}{2} \rho_{\text{trop}}\mathrm{e}^{-\left( \frac{g}{RT_{\text{trop}}}(H-h_{\text{trop}}) \right)} u^2S
\end{equation}

So the total external force on the direction $\vec{e}_r$ is 

\begin{equation}
    w_r=-mg -PA 
\end{equation}

where $A$ is the projected area of the whole plane.

\begin{equation}
    -m(t)\frac{v(t)^2}{r}=-mg- PA +L_u
\end{equation}

\begin{equation}
     -m(t)\frac{v(t)^2}{r}=-mg-  P_{\text{trop}}\mathrm{e}^{-\left( \frac{g}{RT_{\text{trop}}}(H-h_{\text{trop})} \right)}A +\frac{\beta C_L}{2} \rho_{\text{trop}}\mathrm{e}^{-\left( \frac{g}{RT_{\text{trop}}}(H-h_{\text{trop}}) \right)} u^2S
   \end{equation}

notice that $r=R_{\text{earth}}+H$ and $R_{\text{earth}}>> H$, so we approximate $r\approx R_{\text{earth}}$
\begin{equation}
    -m\frac{v^2}{R_{\text{earth}}}= -mg-  P_{\text{trop}}\mathrm{e}^{-\left( \frac{g}{RT_{\text{trop}}}(H-h_{\text{trop})} \right)}A +\frac{\beta C_L}{2} \rho_{\text{trop}}\mathrm{e}^{-\left( \frac{g}{RT_{\text{trop}}}(H-h_{\text{trop}}) \right)} u^2S
\end{equation}

\begin{equation}
\Rightarrow     -m\frac{v^2}{R_{\text{earth}}} +mg= \left(-  P_{\text{trop}}A +\frac{\beta C_L}{2} \rho_{\text{trop}} u^2S\right) \mathrm{e}^{-\left( \frac{g}{RT_{\text{trop}}}(H-h_{\text{trop})} \right)}
\end{equation}

\begin{equation}
  \Rightarrow \frac{-m\frac{v^2}{R_{\text{earth}}} +mg}{\left(-  P_{\text{trop}}A +\frac{\beta C_L}{2} \rho_{\text{trop}} u^2S\right)}=  \mathrm{e}^{-\left( \frac{g}{RT_{\text{trop}}}(H-h_{\text{trop})} \right)}    
\end{equation}
\begin{equation}
  \Rightarrow \log \left(\frac{-m\frac{v^2}{R_{\text{earth}}} +mg}{\left(-  P_{\text{trop}}A +\frac{\beta C_L}{2} \rho_{\text{trop}} u^2S\right)}\right)=  -\left( \frac{g}{RT_{\text{trop}}}(H-h_{\text{trop}}) \right)    
\end{equation}

Isolating $H$ and substituting $u$ we have that
\begin{equation}
    H= - \frac{RT_{\text{trop}}}{g}\log \left(\frac{-m\frac{v^2}{R_{\text{earth}}} +mg}{\left(-  P_{\text{trop}}A +\frac{\beta C_L}{2} \rho_{\text{trop}} |\vec v-\vec W|^2S\right)}\right)+h_{trop}
\end{equation}

 But $C_L$ in literature is given by
\begin{equation}
    C_L= \frac{2mg}{\rho_{\text{trop}} |\vec v-\vec W|^2S}
\end{equation}

\begin{equation}
    \Rightarrow H= - \frac{RT_{\text{trop}}}{g}\log \left(\frac{-m\frac{v^2}{R_{\text{earth}}} +mg}{-  P_{\text{trop}}A +\beta mg }\right)+h_{trop}
\end{equation}

\section{Finding the right altitude to keep a certain speed below the tropopause}

The air density at a certain altitude can be approximated as

\begin{equation}
    \rho= \rho_0 \left(\frac{T_0-\frac{6.5h}{1000}}{T_0}\right)^{-\frac{g}{k_TR}-1}
\end{equation}

and the pressure is 

\begin{equation}
    P=P_0\left(\frac{T_0-\frac{6.5h}{1000}}{T_0}\right)^{-\frac{g}{k_TR}}
\end{equation}

So the total external force on the direction $\vec{e}_r$ is 

\begin{equation}
    w_r=-mg -PA 
\end{equation}

where $A$ is the projected area of the whole plane.

\begin{equation}
    -m(t)\frac{v(t)^2}{r}=-mg+ L_u
\end{equation}

were we dropped the pressure term just to have an estimate altitude

\begin{equation}
  \Rightarrow \frac{-m\frac{v^2}{R_{\text{earth}}} +mg}{\frac{\beta C_L}{2} \rho_0 u^2S}=  \left(\frac{T_0-\frac{6.5h}{1000}}{T_0}\right)^{-\frac{g}{k_TR}-1}    
\end{equation}

\printbibliography

\end{document}